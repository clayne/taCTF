% Created 2021-12-08 Wed 06:51
% Intended LaTeX compiler: xelatex
\documentclass[11pt]{article}
\usepackage{graphicx}
\usepackage{grffile}
\usepackage{longtable}
\usepackage{wrapfig}
\usepackage{rotating}
\usepackage[normalem]{ulem}
\usepackage{amsmath}
\usepackage{textcomp}
\usepackage{amssymb}
\usepackage{capt-of}
\usepackage{hyperref}
\usepackage[top=1.0in,bottom=1.0in,left=0.75in,right=0.75in]{geometry}
\usepackage{siunitx}
\usepackage{booktabs}
\usepackage{minted}
\author{Project By\\ Aditya Nebhrajani\\ \texttt{avn5} \and Overseen By\\ Thomas Quig\\ \and Submitted To\\ Professor Miller}
\usepackage{parskip}
\usepackage[backend=bibtex, style=nature]{biblatex}
\addbibresource{report.bib}
\setminted{linenos=true, breaklines=true, breakanywhere=true}
\date{Fall 2021}
\title{SIGPwny ECE 297 Project}
\hypersetup{
 pdfauthor={},
 pdftitle={SIGPwny ECE 297 Project},
 pdfkeywords={},
 pdfsubject={},
 pdfcreator={Emacs 27.2 (Org mode 9.5)},
 pdflang={English}}
\begin{document}

\maketitle
\begin{center}
\href{https://github.com/nebhrajani-a/taCTF}{\texttt{https://github.com/nebhrajani-a/taCTF}}
\end{center}

\section{Abstract}
\label{sec:orgcff712d}
This report describes an ``automatic'' tool for implementing
a side-channel timing-style attack on compiled binaries. Timing
attacks are often difficult to design around, especially since it may
cause poorer performance. Timing attacks are among the most successful
forms of side-channel attack, with Meltdown\cite{meltdown} and
Spectre\cite{spectre} as recent examples, and initial implementations
of Unix \texttt{login} as classical examples. The tool described by this
report specifically focuses on a simpler kind of side channel: it uses
Valgrind to count the number of instructions executed by a string
comparison function for various inputs, and uses the result to predict
the string.

\section{Background}
\label{sec:org5658d86}

\subsection{Problem Statement}
\label{sec:orgc8b589f}
String comparison is a basic operation in many cryptographic
applications. Consider a C function written the following way\cite{wiki}:

\begin{minted}[]{c}
 bool insecureStringCompare(const void *a, const void *b, size_t length) {
  const char *ca = a, *cb = b;
  for (size_t i = 0; i < length; i++)
    if (ca[i] != cb[i])
      return false;
  return true;
}
\end{minted}

When input strings \texttt{a} and \texttt{b} are identical, all is well. However, if
the strings differ at any character, the function instantly returns,
breaking the \mintinline{c}{for} loop. This makes sense for
efficiency, since we're not wasting time checking the equivalence of
two strings that are obviously unequal. However, this gives us a way to
``brute-force'' one of the strings if the other is held constant: we can
simply try every ASCII character, and if the \mintinline{c}{for}
loop isn't broken, the current character must be the correct one.

Of course, the natural question is, how can we tell from the execution
of a binary file whether the \mintinline{c}{for} loop has been
terminated or not? This is done by counting the number of assembly
instructions executed: more instructions would imply that the
function did not terminate: so we must have the right character.

Preventing such an attack requires that the function check both
strings all the way through, regardless of whether they stop matching
at a certain character. A patched function is thus:

\begin{minted}[]{c}
 bool constantTimeStringCompare(const void *a, const void *b, size_t length) {
  const char *ca = a, *cb = b;
  bool result = true;
  for (size_t i = 0; i < length; i++)
    result &= ca[i] == cb[i];
  return result;
}
\end{minted}

This entirely removes our side channel. Another way that timing
attacks are prevented is by making the program sleep for small and
random amounts of time, or perform some other random operations. This
throws off any tools like ours.

\subsection{Current Tools}
\label{sec:org8a98135}
Given that such a vulnerability is popular among CTF-style reverse
engineering challenges, a tool similar in function to the one we detail
does exist: PinCTF\cite{pinctf}. However, it uses Intel's PIN to count
instructions rather than Valgrind. Intel PIN is notoriously difficult
to install and get working, and has a habit of littering any directory
it's called in with \texttt{inscount.out} files that need to be removed. The
author of this paper could only install PIN on a Garuda Linux virtual
machine, but PinCTF still did not run correctly; which is much more
PIN's fault than the tool's. PinCTF's open GitHub issues are also a
result of PIN's weird behavior. Moreover, our tool's functionality is
a superset of PinCTF's, with automatic length computation and more
options.

In essence, there's a need for a better instruction counting tool
that's plug and play. Valgrind is an excellent instrumentation
framework, and its tool \texttt{callgrind} has an instruction counting
function built-in.\footnote{The tool \texttt{exp-bbv} also has an instruction counting function
built in, but it is ``experimental'' and so we choose to stick with
\texttt{callgrind}.} From testing with multiple binaries, it seems that
this is equally as useful for our purposes as Intel's PIN. While the
absolute value of instructions counted differs between Valgrind and
PIN, the deltas, which are what we care about, agree.

Our tool is called ``taCTF'' (pronounced either ``ta-CTF'' or with each
letter said out loud). The language chosen is Python. Haskell was
considered\footnote{Obligatory ``A monad is just is a monoid in the category of endofunctors'' quote.}, but ultimately dropped due to considerations of ease of
contribution.

The goal while writing this tool was that there should be no
\texttt{requirements.txt} or \emph{any} environment setup required beyond
installing Python 3 and Valgrind (which can be done in a single
command on any good Linux distributions.\footnote{Looking at you, Gentoo.}), and running the
script. We also try to provide some options that make life easier when
doing attacks of this sort, such as reverse-order attacking, known
string prefixes, etc.

\section{Methodology}
\label{sec:org186d130}

Our job is to run Valgrind's \texttt{callgrind} on the binary with various
inputs, parse the output for the instruction count, then use the
instruction count data to reconstruct the string.

\subsection{Parsing \texttt{callgrind} output}
\label{sec:orgac3a7b6}

First, let's look at how \texttt{callgrind} counts instructions:

\begin{minted}[]{text}
$ valgrind --tool=callgrind ./a.out
==43837== Callgrind, a call-graph generating cache profiler
==43837== Copyright (C) 2002-2017, and GNU GPL'd, by Josef Weidendorfer et al.
==43837== Using Valgrind-3.17.0 and LibVEX; rerun with -h for copyright info
==43837== Command: ./a.out
==43837==
==43837== For interactive control, run 'callgrind_control -h'.
==43837==
==43837== Events    : Ir
==43837== Collected : 161590
==43837==
==43837== I   refs:      161,590
\end{minted}

\texttt{Collected : 161590} is the information we need. We can grab it with a
simple regular expression. Also, we'll use Python's \texttt{subprocess}
module (built-in) to call Valgrind:
\begin{minted}[]{python}
def get_instruction_count(test_str, binary_filename):
    with tempfile.NamedTemporaryFile() as tmp:
        command = f'echo "{test_str}" | valgrind --tool=callgrind --callgrind-out-file={tmp.name}\
        {binary_filename}'
        try:
            with subprocess.Popen(command, shell=True, stdout=subprocess.PIPE,
                                  stderr=subprocess.PIPE) as valout:
                valout = valout.stderr.read()
                valout = int(re.findall('Collected : \d+',
                                        valout.decode())[0][12:])
                return valout
        except IndexError:
            pass
\end{minted}

Note that we could use a \texttt{stdin PIPE} instead of \texttt{echo}-ing the string
then piping it via the shell, however, this approach has known bugs in
\texttt{subprocess}\footnote{The documentation recommends using \texttt{subprocess.communicate()}
instead, but this author ran out of time and couldn't play with broken
Python modules. He did add it to the TODO section of the README
though, so perhaps it'll get done someday when he needs to
procrastinate.}. We pass \texttt{IndexError}'s in case the instruction
count isn't found in the output, and the function returns \mintinline{python}{None}.

What's with the temporary file? \texttt{callgrind} writes output files, which
we don't care about (we care about the output on \texttt{STDERR}), so we just
ask \texttt{callgrind} to write it to a temporary file the OS will deal with.
\texttt{tempfile} is also built into Python.

\subsection{Finding the Length}
\label{sec:org6647785}
Assuming the binary instantly quits if the length of the input string
doesn't match that of the secret string, we can find the length by
simply trying every length between 1 and some upper limit. By default,
taCTF sets this upper limit to 35, but it can be changed with a
command line argument. Consider:

\begin{minted}[]{python}
def find_length(binary_filename, maxlen, verbose):
    maximum = -1
    init_counts = get_instruction_count("a", binary_filename)
    length = -1
    for i in range(1, maxlen+1):
        val = get_instruction_count("a"*i, binary_filename)
        if verbose:
            print("a"*i, ":", val)
        if val is not None:
            if val > maximum:
                maximum = val
                length = i
    if init_counts != maximum:
        return length
\end{minted}

If all the lengths happen to be the same, we return \mintinline{python}{None} and move
on. Not all binaries have this weakness.


\subsection{Finding a Single Character}
\label{sec:org5e0be7f}

Given a string and an index, we write a function to get the
instruction count of the string formed by trying character in a given
set at that index. This is trivial:
\begin{minted}[]{python}
def find_char_at(test_str, location, binary_filename, verbose, charset_code):
    if verbose:
        print("Testing: ", make_bold(test_str, location), "at", location)
    charset = get_charset(charset_code)
    maximum = 0
    bestchoice = ""
    for char in charset:
        test = test_str[:location] + char + test_str[location+1:]
        val = get_instruction_count(test, binary_filename)
        if val is not None:
            if val > maximum:
                maximum = val
                bestchoice = test
        if verbose:
            print("    ", char, ":", val)
    return bestchoice
\end{minted}

Naturally, there are probably cooler ways to choose \texttt{bestchoice}. In
the future, the author may add some basic standard-deviation style
statistics to make this choice. For now, a \mintinline{python}{for} loop will suffice.

\subsection{Solve}
\label{sec:org0775b2e}
We can finally write a \texttt{find-string} function that solves our problem:

\begin{minted}[]{python}
def find_string(binary_filename, maxlen=35,
                verbose=False, reverse=False,
                lengthgiven=False, length=0,
                charset_code=0,
                flag_format=""):
    if not lengthgiven:
        length = find_length(binary_filename, maxlen, verbose)
        if length is None:
            print("[taCTF] I couldn't guess the length, sorry. Try -l LENGTH.")
            sys.exit()
    print("Length guess:", length)
    length_diff = length - len(flag_format)
    candidate = flag_format + 'a'*length_diff
    if not reverse:
        for i in range(len(flag_format), length):
            candidate = find_char_at(candidate, i, binary_filename, verbose,
                                     charset_code)
            print(candidate)
    else:
        for i in range(length-1, len(flag_format)-1, -1):
            candidate = find_char_at(candidate, i, binary_filename, verbose,
                                     charset_code)
            print(candidate)
    return candidate
\end{minted}

This function has all the features taCTF has. It simply checks
whether to find the length or if it's been supplied, then calls
\texttt{find\_char\_at} in either left-to-right or right-to-left order as many
times as required.

\subsection{Arguments}
\label{sec:org0fe121e}

This is the easiest part of the code, and is left as an exercise to
the reader.\footnote{I've always wanted to do that. If you aren't in an exercise-y
mood, feel free to look at the code on GitHub.} We instead detail here what our arguments do and how
to use them:

\begin{center}
\begin{tabular}{llll}
\toprule
Argument & Long Form & Function & Default\\
\midrule
\texttt{-h} & \texttt{-{}-help} & Help! & \\
\texttt{-v} & \texttt{-{}-verbose} & Verbose output, with each iteration printed & False\\
\texttt{-r} & \texttt{-{}-reverse} & Try string backwards. & False\\
\texttt{-f} & \texttt{-{}-flag-format} & Known flag format to try. & \\
\texttt{-l} & \texttt{-{}-length} & Length if known. & \\
\texttt{-c} & \texttt{-{}-charset} & Which ASCII character 'set' to try. & 0\\
\texttt{-ml} & \texttt{-{}-max-length} & Maximum length to check till. & 35\\
\bottomrule
\end{tabular}
\end{center}

We also detail our character codes:
\begin{center}
\begin{tabular}{rl}
\toprule
Code & Meaning\\
\midrule
0 & Lowercase, uppercase, digits\\
1 & Lowercase\\
2 & Uppercase\\
3 & Lowercase, uppercase\\
4 & Lowercase, digits\\
\bottomrule
\end{tabular}
\end{center}
Punctuation is included by default in all sets.

\section{Results}
\label{sec:org7ea29b7}

Our goal of making the tool as painless to use as possible\footnote{Finally, we're rid of PIN!} seems to have
succeeded. Let's write a weak C program that has the vulnerability:

\begin{minted}[]{c}
#include <stdio.h>
#include <stdbool.h>
#include <string.h>

#define MAX_INPUT_LEN 100

bool insecure_string_compare(const void *a, const void *b, size_t length) {
  const char *ca = a, *cb = b;
  for (size_t i = 0; i < length; i++)
    if (ca[i] != cb[i])
      return false;
  return true;
}

int main(void)
{
  char flag[] = "sigpwny{flag}";
  char input[MAX_INPUT_LEN];
  fgets(input, MAX_INPUT_LEN, stdin);

  if ((strlen(input) - 1) != strlen(flag)) {
    return 0;
  }
  else {
    insecure_string_compare(flag, input, strlen(flag));
  }

  return 0;
}
\end{minted}

On a sandbox (Arch) Linux machine, all we need to do is:
\begin{minted}[]{text}
$ sudo pacman -S python3 valgrind
$ git clone https://github.com/nebhrajani-a/taCTF
$ cd taCTF
$ gcc test/example.c
$ ./tactf.py ./a.out
\end{minted}

The output is:

\begin{minted}[]{text}
Length guess: 13
saaaaaaaaaaaa
siaaaaaaaaaaa
sigaaaaaaaaaa
sigpaaaaaaaaa
sigpwaaaaaaaa
sigpwnaaaaaaa
sigpwnyaaaaaa
sigpwny{aaaaa
sigpwny{faaaa
sigpwny{flaaa
sigpwny{flaaa
sigpwny{flaga
sigpwny{flag}
\end{minted}

Maybe with some more options:

\begin{minted}[]{text}
$ ./tactf.py ./a.out --flag-format "sigpwny{" -l 13 -c 1
Length guess: 13
sigpwny{faaaa
sigpwny{flaaa
sigpwny{flaaa
sigpwny{flaga
sigpwny{flag}
\end{minted}

Let's also try some other binary that checks the string in reverse
order:

\begin{minted}[]{text}
$ ./tactf.py test/ELF-NoSoftwareBreakpoints -r -l 25
Length guess: 25
aaaaaaaaaaaaaaaaaaaaaaaaS
aaaaaaaaaaaaaaaaaaaaaaakS
aaaaaaaaaaaaaaaaaaaaaackS
aaaaaaaaaaaaaaaaaaaaa0ckS
aaaaaaaaaaaaaaaaaaaar0ckS
aaaaaaaaaaaaaaaaaaa_r0ckS
aaaaaaaaaaaaaaaaaaT_r0ckS
aaaaaaaaaaaaaaaaaNT_r0ckS
aaaaaaaaaaaaaaaaiNT_r0ckS
aaaaaaaaaaaaaaaoiNT_r0ckS
aaaaaaaaaaaaaaPoiNT_r0ckS
aaaaaaaaaaaaakPoiNT_r0ckS
aaaaaaaaaaaaakPoiNT_r0ckS
aaaaaaaaaaa3akPoiNT_r0ckS
aaaaaaaaaar3akPoiNT_r0ckS
aaaaaaaaaBr3akPoiNT_r0ckS
aaaaaaaa_Br3akPoiNT_r0ckS
aaaaaaae_Br3akPoiNT_r0ckS
aaaaaare_Br3akPoiNT_r0ckS
aaaaa@re_Br3akPoiNT_r0ckS
aaaaW@re_Br3akPoiNT_r0ckS
aaadW@re_Br3akPoiNT_r0ckS
aardW@re_Br3akPoiNT_r0ckS
aardW@re_Br3akPoiNT_r0ckS
HardW@re_Br3akPoiNT_r0ckS
\end{minted}



\section{Conclusion}
\label{sec:orgf574fb1}

We've achieved our goal: writing a clean and easy to use instruction
counting tool. There are more features I'd like to add, and at the top
of the list is threading. This, however, is a rather complex endeavor
with diminishing returns, given how Python's global interpreter lock
works. There is perhaps a solution using asynchronous I/O that I'm
currently exploring, and may add. Other things to add include:
\begin{itemize}
\item Better statistics for deciding which letter to choose.
\item Treat program's movement as a tree and allow going back up the tree
and branching in case instruction counts are all equal after a
certain choice.
\item Add PIN support. Yes, PIN is difficult to use, but having support
for PIN gives us another tool to confirm Valgrind's results with.
\item Use \texttt{subprocess.communicate} instead of \texttt{echo} to talk to Valgrind.
This will allow punctuation characters that are currently blocked
due to \texttt{echo}'s limitations and escape character pain.
\end{itemize}

I had a lot of fun writing this small and simple tool, and hopefully
you had as much going through this report. Any suggestions, issues,
comments, are welcome on the GitHub issues tracker of this project.\footnote{AKA, please, tear apart my code.}


\newpage
\printbibliography
\end{document}